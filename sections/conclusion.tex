\section{Conclusion}
\label{sec:conclusion}

As we have demonstrated, a system that continuously spies on the user while they are playing an augmented reality game is feasible to build, especially with the prospect of our employed frameworks being continuously improved. We also showed how a preprocessing on the client allows reducing the amount of data that is actually communicated to the server to a minimum, thus disguising the illicit behavior from the user. Even though a close analysis of the traffic of the application may reveal surprisingly frequent communication with a server, a more elaborate gameplay system incorporating real-time player-vs-player might be able to sell this as necessary as well.

As such, we believe that augmented reality applications may pose a great risk to the consumer. 
This makes it even more important than ever to take care when installing apps and only rely on those which are distributed by trusted sources such as the Google Play Store and Apple's App Store. \\
Additionally, users should always look out for suspicious behavior of apps, for example, if an app uses more permissions than needed for the basic functionalities, transfers lots of data via the Internet or requires a lot of CPU power.
An app is, however, still a black box to the users which will make it impossible for them to clearly separate malicious behavior from badly implemented apps.\\
Unfortunately, to this day there is no perfect protection against the kind of unwanted behavior which is outlined in this paper because a well made malicious app will always be able to hide its intents properly.


% Future Work
- z.B. weitere Daten sammeln und komplexere Nutzerprofile anlegen
- Data Analytics auf den gesammelten Infos
