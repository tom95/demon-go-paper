%----------------------------------------------------------------
% Introduction
%----------------------------------------------------------------
\section{Introduction}
Today's smartphones provide a vast array of capabilities to support their diverse use cases.
Multiple built-in cameras are able to take high-resolution pictures and videos with up to tens of megapixels of combined footage.
Seamless location tracking is enabled via a GPS unit that allows real-time retrieval of the user's geolocation.
Acceleration and gyroscopic sensors provide accurate data about the phone's movements.
With the broad availability of high-speed mobile transmission standards such as LTE, large amounts of data can be sent, enabling upload-intensive tasks such as live video streaming.
In addition, high-end smartphones are equipped with high-performance, multicore CPUs and a sizeable amount of RAM, bringing with them veritable processing power on their own.
With the introduction of Augmented Reality (AR) frameworks, we see these capabilities put to use to deduce a 3D point cloud reflection of the user's environment through computationally complex analysis of the camera feed and inertial sensors.
% - features of today's smartphones 
%  - cameras able to take high resolution pictures/videos
%  - geolocation via GPS
%  - ability to use high speed internet (where available) -> send high amounts of data/video streaming possible
%  - high processing power, multiple cores, quite some RAM
%  - allows complex calculations, analysis of video data, e.g. in context of AR

The first wildly successful application to include these AR features was Pokémon GO, a popular game about catching Pokémon in the real world~\cite{pkmgo}.
As of May 2018, it has been downloaded more than 800 million times, and it had around 45 million daily active users in its prime time~\cite{PkmDownloads}.
% - Pokemon-Go
% - popular game about catching Pokemon in the real world
% - around 800 million downloads (30.5.18, https://www.playm.de/2018/05/pokemon-go-35-403807/)
The app utilizes geolocation tracking to identify Pokémon, events and special places near the user's location, while an AR environment is created to interact with Pokémon.
To play the game, the user necessarily requires an internet connection to connect to an external server which controls the game flow and receives player data.
Players noted the high battery consumption posed by the requirements of the game but adopted by bringing external battery packs~\cite{PkmBattery}.
% - uses location of user to identify Pokemon/Events/special places near him
% - able to use AR to interact with Pokemon in real world 
% - (always) connected to the internet and able to transfer data, no opt-out

We provide an example of how such an application could misuse the given access permissions, especially because access to sensitive data such as the camera feed is expected by the user.
% - possible to create very detailed profiles of players
% - tracking of location and movement (further analysis could discover certain habits/patterns)
This unchecked access allows the computational analysis of everything the user points their camera at.
This may include textual information, faces of persons in the vicinity, or discernible brand logos.
While highly unethical, this analysis could be used for a wide range of use cases, including espionage, preparation of or even automated data theft, or simply targeted advertisements.
% - analysis of everything where persons points their camera at: texts, faces, brands
% - can be used for espionage, preparation of crimes or simply targeted advertisements
Our prototype aims to showcase the potential dangers that the improvident usage of AR-applications can bring, focusing on the demo cases of espionage of sensitive data as well as brand label recognition to create a profile of the user's consumption behavior.
We try to keep the showcase close to real-world applicability, dealing with concerns such as scalability and limited data transfer volumes.

% - hence we created a demo showcase to reveal the potential dangers that the improvident usage of AR-apps can bring
