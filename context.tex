%----------------------------------------------------------------
% Lehrstuhlkontext: 
% Diesen Abschnitt (in der deutschen bzw. englischen Fassung) übernehmen 
%----------------------------------------------------------------

%\section{Kontext}
%\label{sec:Kontext}
%Die Betreuung im Rahmen der Seminartätigkeit erfolgte durch das Fachgebiet für Computergrafische Systeme, dessen Forschungsschwerpunkt die Prozessierung, Abbildung und interaktive Visualisierung massiver raumzeitlicher \cite{Oehlke2015,Buschmann2015,Buschmann2014,Maass2006} sowie abstrakter, hochdimensionaler Daten \cite{Limberger2017,Limberger2016,Wuerfel2015} ist. Dies beinhaltet neben neuartigen Algorithmen \cite{RichterKyprianidis2013,RichterBehrens2013,Glander2012}, Rendering-Techniken \cite{Semmo2016,Pasewaldt2014,Maass2006a,Doellner2005} und Interaktions-Metaphern \cite{Semmo2016a,Scheibel2016,Semmo2014} auch effiziente Datenstrukturen \cite{Scheibel2017,Richter2015} und Systemarchitekturen \cite{Klimke2014,Trapp2012,Klimke2010}, die anhand von real-weltlicher Datensätze und Anwendungsszenarien  \cite{Discher2016,Trapp2015,Engel2012} evaluiert werden. 

\section{Context}
\label{sec:Context}
[[[ This course project was supervised by the Computer Graphics Systems Group whose main research interest includes the processing, mapping, and interactive visualization of massive spatio-temporal information \cite{Oehlke2015,Buschmann2015,Buschmann2014,Maass2006} and abstract high-dimensional information \cite{Limberger2017,Limberger2016,Wuerfel2015}. In particular, this comprises novel algorithms \cite{RichterKyprianidis2013,RichterBehrens2013,Glander2012}, rendering techniques \cite{Semmo2016,Pasewaldt2014,Maass2006a,Doellner2005}, and interaction metaphors \cite{Semmo2016a,Scheibel2016,Semmo2014}, as well as efficient data structures \cite{Scheibel2017,Richter2015} and system architectures \cite{Klimke2014,Trapp2012,Klimke2010} which are evaluated based on real-world data sets and application scenarios \cite{Discher2016,Trapp2015,Engel2012}. ]]]
